%%%%%%%%%%%%%%%%%%%%%%%%%%%%%%%%%%%%%%%%%%%%%%%%%%%%%%%%%%%%%%%%%%%%%%%%
%% Customizações do abnTeX2 (http://abnTeX2.googlecode.com)           %%
%% para a Universidade Estadual do Ceara - UECE                       %%
%%                                                                    %%
%% This work may be distributed and/or modified under the             %% 
%% conditions of the LaTeX Project Public License, either version 1.3 %%
%% of this license or (at your option) any later version.             %%
%% The latest version of this license is in                           %%
%%   http://www.latex-project.org/lppl.txt                            %%
%% and version 1.3 or later is part of all distributions of LaTeX     %%
%% version 2005/12/01 or later.                                       %%
%%                                                                    %%
%% This work has the LPPL maintenance status `maintained'.            %%
%%                                                                    %%
%% The Current Maintainer of this work is Thiago Nascimento           %%
%%                                                                    %%
%% Project available on: https://github.com/thiagodnf/uecetex2        %%
%%                                                                    %%
%% Further information about abnTeX2                                  %%
%% are available on http://abntex2.googlecode.com/                    %%
%%                                                                    %%
%%%%%%%%%%%%%%%%%%%%%%%%%%%%%%%%%%%%%%%%%%%%%%%%%%%%%%%%%%%%%%%%%%%%%%%%

\documentclass[        
    a4paper,          % Tamanho da folha A4
    12pt,             % Tamanho da fonte 12pt
    chapter=TITLE,    % Todos os capitulos devem ter caixa alta
    section=TITLE,    % Todas as secoes devem ter caixa alta
    oneside,          % Usada para impressao em apenas uma face do papel
    english,          % Hifenizacoes em ingles
    spanish,          % Hifenizacoes em espanhol
    brazil            % Ultimo idioma eh o idioma padrao do documento
]{abntex2}


% Importações de pacotes
\usepackage[utf8]{inputenc}                         % Acentuação direta
\usepackage[T1]{fontenc}                            % Codificação da fonte em 8 bits
\usepackage{graphicx}                               % Inserir figuras
\usepackage{amsfonts, amssymb, amsmath}             % Fonte e símbolos matemáticos
\usepackage{booktabs}                               % Comandos para tabelas
\usepackage{verbatim}                               % Texto é interpretado como escrito no documento
\usepackage{multirow, array}                        % Múltiplas linhas e colunas em tabelas
\usepackage{indentfirst}                            % Endenta o primeiro parágrafo de cada seção.
\usepackage{microtype}                              % Para melhorias de justificação?
\usepackage[portuguese,ruled,lined]{algorithm2e}    % Escrever algoritmos
\usepackage{algorithmic}                            % Criar Algoritmos  
%\usepackage{float}                                  % Utilizado para criação de floats
\usepackage{amsgen}
\usepackage{lipsum}                                 % Usar a simulação de texto Lorem Ipsum
%\usepackage{titlesec}                               % Permite alterar os títulos do documento
\usepackage{tocloft}                                % Permite alterar a formatação do Sumário
\usepackage{etoolbox}                               % Usado para alterar a fonte da Section no Sumário
\usepackage[nogroupskip,nonumberlist,acronym]{glossaries}                % Permite fazer o glossario
\usepackage{caption}                                % Altera o comportamento da tag caption,nb          
\usepackage[alf, abnt-emphasize=bf, bibjustif, recuo=0cm, abnt-etal-cite=2, abnt-etal-list=0]{abntex2cite}  % Citações padrão ABNT
%\usepackage[bottom]{footmisc}                      % Mantém as notas de rodapé sempre na mesma posição
\usepackage{helvet}
\renewcommand{\familydefault}{\sfdefault}                                % Usa a fonte Arial
\usepackage{mathptmx}                               % Usa a fonte Times New Roman										
%\usepackage{lmodern}                               % Usa a fonte Latin Modern
%\usepackage{subfig}                                % Posicionamento de figuras
%\usepackage{scalefnt}                              % Permite redimensionar tamanho da fonte
%\usepackage{color, colortbl}                       % Comandos de cores
%\usepackage{lscape}                                % Permite páginas em modo "paisagem"
%\usepackage{ae, aecompl}                           % Fontes de alta qualidade
%\usepackage{picinpar}                              % Dispor imagens em parágrafos
%\usepackage{latexsym}                              % Símbolos matemáticos
%\usepackage{upgreek}                               % Fonte letras gregas
\usepackage{appendix}                               % Gerar o apendice no final do documento
\usepackage{paracol}                                % Criar paragrafos sem identacao
\usepackage{lib/uecetex2}		                    % Biblioteca com as normas da UECE para trabalhos academicos
\usepackage{pdfpages}                               % Incluir pdf no documento
\usepackage{amsmath}                                % Usar equacoes matematicas
\usepackage{colortbl}
% Organiza e gera a lista de abreviaturas, simbolos e glossario
\makeglossaries

% Gera o Indice do documento
\makeindex
