In this on the radio waves: popular culture, peasants and the Basic Education Movement we analyze the participation of peasants of the Brazilian northeastern region in the Basic Education Movement. The focus of this thesis is to demonstrate how the labors involved with broadcast schools have elaborated actions for maintaining and spreading the schools in their communities, in order to achieve the necessary means to improve their way of life. Peasants of the Basic Education Movement have been coadjuvant of the modernizing catholic proposition of the early 1960s, by means of quite peculiar political and cultural representations. Some of these representations were: a meaning for the school, a role for the union and for the political participation, precepts of the land use rights and labor rights, and the multiple meanings of the radio as a mass communication, information and leisure medium. This study intends to stress that the actions – and the political enrollment – of the northeastern peasant could not ever be separated from the modernizing process. The connection can be observed in different social movements of the period, such as the Basic Education Movement, rural unions, the Catholic Agrarian Youth and the MCP. In this sense, we consider that, if the Brazilian modernization was a guideline for the institutions, political organisms and parties for the social movement, such a modernization was a guideline of demands based on elements of material life. Those elements included, by that time, the agrarian reform, the educational issue and labor urgencies.

% Separe as Keywords por ponto
\keywords{Adult education. Community schools. Peasants. Popular culture}